% LaTeX resume using res.cls
\documentclass[margin]{res}
%\usepackage{helvetica} % uses helvetica postscript font (download helvetica.sty)
%\usepackage{newcent}   % uses new century schoolbook postscript font
\usepackage[colorlinks = true,
            linkcolor = blue,
            urlcolor  = blue,
            citecolor = blue,
            anchorcolor = blue]{hyperref}
\setlength{\textwidth}{5in} % set width of text portion
% \vspace*{-6mm}

\begin{document}
\vspace*{-2.0cm}
% Center the name over the entire width of resume:
 \moveleft.5\hoffset\centerline{\Large\bf Gaurav Kulkarni}
 \moveleft.5\hoffset\centerline{\url{http://gauravkulkarni.herokuapp.com}}
 %\moveleft.5\hoffset\centerline{Github : \url{https://github.com/gauravkulkarni96}}
% \moveleft.5\hoffset\centerline{LinkedIn : \url{https://in.linkedin.com/in/gauravkulkarni96}}
% Draw a horizontal line the whole width of resume:
 \moveleft\hoffset\vbox{\hrule width\resumewidth height 1pt}\smallskip
% address begins here
% Again, the address lines must be centered over entire width of resume:
% \moveleft.5\hoffset\centerline{The LNM Institute of Information Technology}
% \moveleft.5\hoffset\centerline{Sumel, Via-Jamdoli, Jaipur 302031. India}
 \moveleft.5\hoffset\centerline{E-mail : gauravkulkarni96@gmail.com}
 \moveleft.5\hoffset\centerline{Phone : +91-774-266-9667}
 \moveleft.5\hoffset\centerline{Github : \url{https://github.com/gauravkulkarni96}}
 \moveleft.5\hoffset\centerline{LinkedIn : \url{https://in.linkedin.com/in/gauravkulkarni96}}



\begin{resume}

\section{EDUCATION}

\begin{tabular}{|c|c|c|c|}

\hline
\textbf{Degree} & \textbf{Year} & {Institution/Board} & {CPI/\%} \\
\hline
B.Tech (CSE) & 2015-2019* & The LNM Institute of Information Technology, Jaipur & 7.18/10 \\
\hline
XII & 2015 & Delhi Public School, Bhilai (CBSE) & 92\% \\
\hline
X & 2013 & Delhi Public School, Bhilai (CBSE) & 10/10 \\
\hline
\end{tabular}
*expected

\section{WORK\\EXPERIENCE} {\textbf{Curieo - Software Development Intern \hfill March `17 - Present\\}}
Leading the development of Programming module for Curieo's Educational platform. The module aims at creating a student friendly platform for a better learning experience relevant to programming courses.\\

\section{PROJECTS\\(WEB DEVELOPMENT)}
  {\textbf{Public Blog (RESTful)}} (\href{https://github.com/gauravkulkarni96/blog}{Github}) (\href{https://gauravkulkarni.herokuapp.com/blog}{Live Project})\\
  Public blogging website for personal use. Provides all the functionality of a blog for both, the blogger as well as the reader including creation, deletion and updation of posts, searching by keyword and also its use as an API (RESTful)\\
Technologies used :
   \begin{itemize}  \itemsep -2pt %reduce space between items
   \item Backend : Python, Django, Django REST Framework
   \item Database : PostgreSQL
   \item Frontend : HTML, CSS, Bootstrap
   \end{itemize}

   {\textbf{MicroBlog - Private diary}} (\href{https://github.com/gauravkulkarni96/MicroBlog}{Github}) (\href{https://microblog96.herokuapp.com}{Live Project})\\
    MicroBlog provides a platform for maintaining a private diary/journal. The user can register/login to create entries in the diary and the entries of that specific user are the only ones he has access to.\\
Technologies used :
    \begin{itemize} \itemsep -2pt
     \item Backend : Python, Flask
     \item Database : MySQL
     \item Frontend : HTML, CSS, Bootstrap
     \end{itemize}

   {\textbf{MVP Landing}} (\href{https://github.com/gauravkulkarni96/sample-django-project}{Github})\\
    Basic design for the landing page of a Minimum Viable Product (MVP) including features Login, Signup, Subscriptions, Email verifications etc.\\
Technologies used :
    \begin{itemize} \itemsep -2pt
     \item Backend : Python, Django, Django Registration Redux
     \item Database : SQLite3
     \item Frontend : HTML, CSS, Bootstrap
     \end{itemize}

\section{PROJECTS\\(ACADEMIC)}
{\textbf{Input data validation}} (\href{https://github.com/gauravkulkarni96/data-validation}{Github})\\
    Created a library of functions for input data validation. It includes validation of name, phone number, institute roll number, etc.\\
Technologies used :
    \begin{itemize} \itemsep -2pt
     \item C
     \end{itemize}
\newpage
   {\textbf{Graphical comparison of Sorting Algorithms}} (\href{https://github.com/gauravkulkarni96/compare-sorting-algorithms}{Github})\\
    Compared different sorting algorithms on the basis of time and steps required by an algorithm to sort a randomly generated data set. Algorithms taken under observation : Bubble sort, Merge sort, Selection sort, Quick sort and Insertion sort.\\
Technologies used :
    \begin{itemize} \itemsep -2pt
     \item C, Shell script for data set generation and time and steps observation.
     \item \LaTeX, Beamer for report generation and presentation.
     \item GNU Plot for plotting of observed statistical data.
     \end{itemize}
 \textbf{More projects on \href{https://github.com/gauravkulkarni96}{Github} account.}


\section{TECHNICAL \\ SKILLS} {\textbf{Languages}:} Python, C, C++, Java\\
	{\textbf{Web Development(Backend)}:} Django, Flask, Django REST Framework\\
    {\textbf{Web Development(Frontend)}:} HTML, CSS, Bootstrap\\
    {\textbf{Databases}:} MySQL, PostgreSQL, SQLite\\
    {\textbf{Scripting}:} Shell\\
    {\textbf{Others}:} Command Line\\
    {\textbf{Platforms}:} Linux



\section{COURSES \\ COMPLETED} {\textbf{Computer Science and Engineering}}\\
Computer Programming, Discrete Mathematical Structures, IT Workshop, Data Structures and Algorithms, Digital Circuits and Systems\\

\section{ACHIEVEME-\\NTS}
    Secured $1^{st}$ position in district level Programming Contest.\\
    Secured $1^{st}$ position in school level Programming Contest.\\

\section{POSITIONS OF\\ RESPONSIBIL-\\ITY}
    Teaching Assistant (TA) for Computer Programming course. \hfill June `16 - Dec `16\\
    Active core-member(Web development div.) of Computer club. \hfill July `16 - Present\\
    Organizing team member of LNMHacks (hackathon). \hfill July `16 - Present\\
	Active member of Literary Club. \hfill July `15 - Present\\


\begin{center}
  \begin{footnotesize}
    Last updated: \today \\
  \end{footnotesize}
\end{center}

\end{resume}
\end{document}
